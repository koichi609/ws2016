\documentclass[twocolumn]{jarticle}

%Confirmation Number: 	226
%Submission Passcode: 	226X-C6J8D6B5B7

\usepackage{NLP}
\usepackage[dvipdfmx]{graphicx}

\title{\textbf{テキストマイニングシンポジウムでの発表内容と言語処理技術}}

\author{
竹内 孔一 \and  金山 博 \and 市瀬 眞 \and 榊 剛史 \and
渡辺 靖彦 \and  東中竜一郎 \and 嶋田 和孝   }
\begin{tabular}{c@{\ \ \ \ }cc}
岡山大学大学院 & 日本IBM & NTTドコモ  \\
ホットリンク & 龍谷大 & NTT & 九工大
\vspace{-4ex}
\end{tabular}
\date{\texttt{koichi@cl.cs.okayama-u.ac.jp}}

\begin{document}
\maketitle


\section{はじめに} 
言語理解とコミュニケーション研究会では2011年からテキ
ストマイニングシンポジウムを開催し現在8回開催している.そこで行われた
研究発表からどのようなテーマが議論され,どのような手法が提案され,どの
ような問題が残り,言語処理がどのように役立つかを論じたい.全てを紹介し
きれないが,コールセンター,事故事例,医療,旅行情報,金融情報,経営情
報といったテキストが対象となり,こうした広範な分野に対して,それぞれに
特徴の異なる手法が提案されて状況を報告する.これにより言語処理の実応用
研究の可能性について議論する.




テキストマイニングシンポジウムで発表されてきた内容の中で,学術から見て特徴的な
ものを取り上げ,言語処理技術の発展に参考になればと思い,5年分の発表の中から
取り上げて説明する.また事例を取り上げた後,テキストマイニング全てに共通した
言語処理の置かれている位置付けを確認し,実社会の要求に応える言語処理について
考える.


いくつかを取り上げ,対象とするテキストとどういう問題があるか.

sec: テキストマイニングの課題の特徴.cite{nasukawa本}
価値あるテキスト,価値があることをしる (2012 nasukawa)

問題の特徴として,検索と大きく異なり,なにを取り出して良いか分からないことにある.
少しテキスト,キーワードどういうテキストが入ってるか予測し,集約したもの
を見ながら,検索者が気がつかなくては行けない.ただそうした問題だけで無く,
鳥題したいモノがはっかりしてもパターンが書けないモノもある.テキストとテキスト集合に
価値があり,あるものを鳥代替ことには変わらない.
分野依存,問題依存で必要となる処理がまるで異なってしまう.逆にいうと必要される
確立されていない(どころか取り上げられていない)手法があることになる.
本論ではこうした内容を取り上げて,実社会が要求する言語処理,それによりこんな価値が
あるということを明らかにする.

sec: テキストマイニング具体例
学術的な発表と企業デモ,討論スタイル,さまざまな発表があった.
実タスクにおける問題,学術的な発表としては問題を仮定してそれに対する手法の展開.
学術も手法,辞書構築や言語分析などもあるが,分野非依存手法
本稿では,取り上げられた話題のうち,どんな課題でどういう情報を取り出す必要があるか
また取り出したものがどういう価値があったかを展開し,言語処理の位置付けを行いたい.
テキストから取りたいものが,定義がかなり難しく,パターン化が絶望的で有り,
そのなかで,手法として確立すべき要素を提供できるのではと考えられるためである.
鳥題したいて帰すとは考えさせられるためである.


1) 企業の業績・活動にまつわるテキストマイニング3件
2011 - 2件  和泉,羽室
2012 2nd 大熊-課題抽出
2013 (3rd オリンピック) 3件 坂地,中山,西沢
  ニューすセンチメント 薄い( 2014の京都) 2回
   有価証券利益率   廣川
2013 大森 電機業界 
2014-09 企業Webページを対象とした ジョブマッチ 坂地 
2015 大森寛文(清泉女学院短大

2) 医療介護福祉
2014 9の 3件  山下, 介護(産総研),介護(産総研)
2015-9  1:  長期在院の推定 山下

3) 政策にかかわる意見集約2件 
 2012 木村 
 2016 大規模パブコメ 愛知工業大

4) 高齢者スキルマッチ
5) 未来予測情報抽出 ○島岡聖世・

重要さが違う.テキストから価値ある情報がとれるかどうか.
自分が理解する,知る,部署内での説得,上層説得のための材料としてまで想定される

2種類の課題: 取りたい形式がはっきりしているもの.企業短観から業績に良い悪い
テキストからなにもわからない.おおざっぱなクラスタリング,では意味が無く,
細かすぎると読むのが間に合わない集計も無理

  言語処理における実タスクの1つ応用.意見が取り出した文に価値がある.
  数を集め,傾向がかわる.比較により何が問題か指し示す

 学術的に論文を書くのは難しい.実タスクからもツールを入れてもうまくいかない.
 那須川jsaiぐらい.画期的な技術は辞書構築装置.部分問題で新たな知見が得られれば
 学術と企業研究が結びつき回り出すはずだ.よってどんなデータとどんな問題意識があるか
 ここで取り上げたい(こちらは学術なので).是非参加してほんとに企業の方のニーズを
知って役立つ研究テータ(実社会の要求に即した研究テーマの構築を)

大規模テキストデータからどのように価値あるデータを取り出したか実例がほしい
その実例を発表の中からいくつか取り上げてみたい.

テーマ他によるまとめ
1. 金融,テーマ

取り出し対友の
 陸上競技,

電機業界の分析
政治パブ※の分析

リハビリ患者
医療介護
- 申し送り電子化
学術: 定量的化によるへりの証明
手法: 単語共起ネットワーク
      人手により問い合わせ


* 全体象からの位置付け
  情報源: 自社データ,web上open,ソーシャル
問題点: ソーシャル用のテキスト解析


分野依存: 医療系: 検査値などの複合体+ 医者のテキスト
目標: さまり,要約,違いに対する証明が期待される
必要とされる技術: 

分野: 金融
目標: 予測
必要とされる技術: 

2014 9
酒井坂地 企業webページ検索クエリのタグ推定
 「自然言語処理」でマッチしない.
key: 日経の業績発表記事

ライフイベント属性獲得 増山

2014 2
企業業績評価 (神戸情報大) 
ニュースがどう解釈されたか,株価の変動

2015 2

2012 8 (2nd)
地方要求 国会議事録系 2012-10 荒木

文の種類: 営業日報: 
目的: 課題 営業上の課題
手法: SVM, n-gram 文特徴,語彙リソース,PMI  (文書分類)

那須川: 個々のテキストで生えられない知見をえる Heart

電力中央研究所
PCヘルプデスク障害対応記録

高齢者jobマッチンク


* 言語処理の位置付け
上記いろいろ提示したが,言語処理の位置付けにおきづきに成っただろうか.
言語処理はテキストマイニングでは中心ではないということ
テキスト発見のための検索,表示,テキストで説得するための集約装置(コントローラぶるな)

分野非依存の類義語でまとめてしまうと,重要な情報を見落とす可能性があるため基本単語
が重要である.「夫」食べる,差し入れ

  テキストから価値ある情報を取り出す
  ======


いくつかを取り上げ,対象とするテキストとどういう問題があるか.

sec: テキストマイニングとは何か.cite{nasukawa本}
  言語処理における実タスクの1つ応用.意見が取り出した文に価値がある.
  数を集め,傾向がかわる.比較により何が問題か指し示す

 学術的に論文を書くのは難しい.実タスクからもツールを入れてもうまくいかない.
 那須川jsaiぐらい.画期的な技術は辞書構築装置.部分問題で新たな知見が得られれば
 学術と企業研究が結びつき回り出すはずだ.よってどんなデータとどんな問題意識があるか
 ここで取り上げたい(こちらは学術なので).是非参加してほんとに企業の方のニーズを
知って役立つ研究テータ(実社会の要求に即した研究テーマの構築を)

テーマ他によるまとめ
1. 金融,テーマ

取り出し対友の
 陸上競技,

電機業界の分析
政治パブ※の分析
リハビリ患者


* 全体象からの位置付け
  情報源: 自社データ,web上open,ソーシャル
問題点: ソーシャル用のテキスト解析


分野依存: 医療系: 検査値などの複合体+ 医者のテキスト
目標: さまり,要約,違いに対する証明が期待される
必要とされる技術: 

分野: 金融
目標: 予測
必要とされる技術: 


* 言語処理の位置付け
上記いろいろ提示したが,言語処理の位置付けにおきづきに成っただろうか.
言語処理はテキストマイニングでは中心ではないということ
テキスト発見のための検索,表示,テキストで説得するための集約装置(コントローラぶるな)


\bibliographystyle{jplain}
\bibliography{allu8,my-resultsu8}
\end{document} 

